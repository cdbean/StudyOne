\section{Classroom Study Settings}

The context for this study was an undergraduate course with an enrollment of 98 students in two class sections, in an intelligence training program in a US university. The program was designed to train students to become professional intelligence analysts. All the students were majoring in the program. 73 students consented to our study protocol and agreed that their data be collected and analyzed. They were randomly assigned to small teams of three (two teams were made of three members) making up 25 teams in total. Most (75\%) of them are in the third academic year (3.05 years in average), indicating that participants in our study have relatively advanced experience and knowledge in intelligence analysis. Participants' age ranges from 19 to 28 (20.3 in average). 77\% of the participants are male. The other 25 students opted out of the study (did not submit the consent form), thus their data was excluded from the analysis.

A key requirement of the course is to emphasize hands-on practice on team-based intelligence analysis. During the first nine weeks, students were taught several structured analytic techniques, such as ACH and network analysis, and instructed to utilize PARC ACH and IBM Analyst's Notebook to solve exercise projects.

Students used CAnalytics for a one-week-long project. The project task is to find the robber(s) in seven hypothetical bank robberies fabricated by the course instructor. Students were provided with documents describing information from witnesses, video records, and news media. Students needed to not only identify the key entities in the documents but also infer possible links between entities and across robberies. The result of the task is open-ended, which means that there is no single answer to the task. Participants must make full use of the provided materials and make their best hypotheses based on evidence. This situation is very similar to real world settings and adds to the complexity of the task. The study is administered in the ninth week of the course. Before then students had learned all analytic theories and strategies (e.g. ACH, IEW) taught in that course and had hands-on experience on some state-of-the-art tools, such as IBM Analyst's Notebook and PARC ACH. The instructor told the students that a minimum of 6 hours was expected to complete the project, including in-class and outside-class work. By the end students were required to submit a group report, describing their findings and supporting evidence.

Students were given a tutorial on CAnalytics a week before the project began. One of the authors first walked students through features of CAnalytics and then let students accomplish a small case analysis themselves to ensure everyone was comfortable with the tool. When the project started, one author was always available to help with any technical issues. Although students were encouraged to make full use of CAnalytics, to ensure a naturalist environment they were always free to apply any other tools that they believed to be useful to their work. . 

In total, 73 students were enrolled in the study from two course sessions. Students were assigned randomly by the instructor into small groups of three members (two groups were made of two students), therefore 25 groups. The rest of the students in the course (25 students) opted out of the study. They employed analogous methods (Analysts' Notebook, ACH, etc.) for the project, but their data is excluded in the analysis.



Our collected data includes three aspects: 1. a post-study questionnaire, 2. system log, and 3. team report. In the questionnaire, we includes several 7-index questions measuring individual's self-reported awareness \cite{Convertino2011}, perceived performance \cite{Goyal2014}, and usability (using SUS \cite{}). The questionnaire also includes several open-ended questions asking how the tool helped or impeded their work. 

The system log is carefully designed so that we capture high-level user activities instead of mouse events and keyboard strokes. For example, we log when the user creates, updates, reads, and deletes an entity. We also log which tool the user is working with. Table~\ref{table:log_schema} shows the schema we use for the log system.  

Although there is no single answer to the task problem, there exist several pieces of key evidence that suggest links across different robbery cases. Together with the course instructor (also the creator of the project task), we designed a rubric for grading. For example, whether they captured that Case 3 and Case 5 is much likely to be committed by the same group of robbers; whether they captured that a ``copycat'' in Case 6? The team get one point when they discussed that hypothesis, and get one point when they provide reasonable evidence.

As suggested by prior works \cite{Neale2004,Goyal2014}, we employed a battery of measures in the study. We administrated two mid-surveys, each at the end of the class, immediately after they finished working in class. The mid surveys measure their self-reported awareness, including their awareness of what teammates have done in the past, what they are doing, and what they are going to do, as well as their awareness of the overall group goals and plans. During class, 15 groups were randomly selected to be observed by research assistants. Assistants were required to only silently observe and take notes and not to interfere participants. At the end of the project, we administrated a post-survey, repeatedly measuring their self-reported awareness. The post survey also includes several open-ended questions about their use experience and difficulty in the process.



% demographic information