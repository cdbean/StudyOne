\begin{abstract}
Intelligence analysis involves dynamic modeling and exploration of complex information spaces, and is fundamentally collaborative at any non-trivial 
scale. That said, the tools supporting intelligence analysis either do not 
support such dynamic workflows, or they do not support collaboration at all. 
Along these lines, we developed \emph{CAnalytics} to address these two critical 
challenges by: enabling interleaving of data modeling and analysis within a single 
workspace; and supporting collaboration through shared annotation and increased 
activity awareness. This paper reports a classroom study of students training to 
become intelligence analysts and examines their tool usage over multiple usage 
sessions. We find that students leveraged our tool to interleave the different 
stages of analysis and took advantage of the collaboration features. We also note the 
distinctions among different analytic and collaborative strategies. The paper concludes with 
design implications for intelligence analysis as well as other collaborative information analysis tools.
\end{abstract}


% in discussion talk about broader usage of the tool