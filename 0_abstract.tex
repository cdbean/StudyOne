% abstract

Collaborative information analysis involves strategic analysis and representation of complex information space through synchronous and asynchronous team interactions over extended time periods. Chances are rare to observe how technology mediates team analysis because such complex scenarios are difficult to model in lab studies, and analysts are often limited to access in real world. Classroom study provides a testbed in which students are trained to become professional analysts and course projects are designed to simulate real world tasks. We deployed our tool, a collaborative visual analytic tool informed by earlier work, in class and observed the usage of the tool by teams in their project. Particularly, this paper reports how participants used the tool in evidence collection, evidence schematization, hypothesis generation, and overall coordination, four critical parts in the task of collaborative information analysis. The paper ends with discussion over several theoretical and design implications.
%
%Collaborative information analysis typically involves representation and analysis of complex information space through synchronous and asynchronous team interactions over extended time periods. It is important to examine, understand, and provide effective tools and environments for these long-term, real world CSCW interactions. We have developed a tool, CAnalytics, to support collaborative information analysis through coordinated multiple views, to raise user awareness through real time sharing and other mechanisms, and to facilitate communication with an enhanced messaging tool. We deployed CAnalytics in an intelligence analysis course, in which 76 analysts-in-training used our tool for a one-week-long project. We report users' experience in this paper. Participants found the tool facilitated their teamwork and increased their engagement, but they also experienced breakdowns. The paper ended with discussion over design implications we derived from the field study.