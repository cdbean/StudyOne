\section{Related work}\label{related-work}

\subsection{Intelligence analysis workflow}

Pirolli and Card's Think Loop Model \cite{Pirolli2005} is a widely used model of intelligence analysis. The model describes the process of information foraging and sensemaking in which raw evidence is successively modeled, filtered, and synthesized into a best hypothesis. The model is a bottom-up process of structure building, but also includes a local feedback loop at each stage. Thus, analysts can reconsider propositions in the evidence file, asking how they are related, or a given hypothesis, asking what schemata it rests upon.

Empirical studies of information analysis suggest that the iterative looping can have a wider scope than is obvious in the Pirolli and Card's model \cite{Pirolli2005}. For example, Chin et al. \cite{Chin2009} observed five professional intelligence analysts working both individually and as a team. They found that analysts often need to review the original documents even at advanced stages of analysis. The scope of these reconsiderations is not consistent with the local feedback architecture of the Think Loop Model. It seems more consistent with a parallel or multi-phased model \cite{Wheaton2011} in which structure building occurs at a variety of levels in parallel. Similar findings were reported by other empirical studies \cite{Isenberg2008b,Kang2011,Herrmann2013a}.

However, tools supporting intelligence analysis are often designed targeted at a single phase of activity, and thus not supporting the whole workflow. For example, research efforts have been made to understand collaborative modeling \cite{Prilla2013}, but little is understood how such models could be extended to contribute to analysis.  In contrast, many analytic tools assume data model has been
finalized and ready to be visually analyzed, and affords no utility
to construct or refine data models. As Ware termed as \emph{``asymmetry in
data rates''} \cite[p.382]{Ware2012}, analytic tools emphasized data
flowing from systems to users far more than from users to systems.
Functionalities are mostly designed to adjust data representation
rather than modeling, which are in fact equally important in intelligence analysis. A report from a workshop of professional intelligence analysts listed ``dynamic data processing and visualization'' as one of top requirements in computational support \cite{Badalamente2005}, and Kang et al. \cite{Kang2011} made another call from findings of their empirical study.

\subsection{Collaboration and awareness in intelligence analysis}

Collaboration is critical in intelligence analysis. The intelligence community has called for collaboration, and indeed closer collaboration, as opposed to merely coordinating draft products by the end \cite{Vision2015}. Empirical studies \cite{Chin2009,Kang2011} reported that analysts' practice was fundamentally collaborative.

However, computational tools supporting intelligence analysis either do not do support collaboration, or they do not integrate serious analytic support. On one hand, not most analytic tools (e.g.~Analyst's Notebook and PARC ACH) are designed for single user only, with only a few exceptions (e.g.~Te@mACH \cite{Globalytica2017}). On the other hand, collaboration tools only support general coordination between teams and agencies, without analytic functionalities, as Treverton \cite{Treverton2016} reviewed collaboration tools currently used in the intelligence community. For example, Intellipedia \cite{Intelink2017} is a wiki platform for sharing of intelligence reports and documents, yet it does not support data modeling and analysis at all.

A key enabler for effective collaboration is \emph{activity awareness}, defined as
team's awareness of its own sustained collaborative activity
\cite{Carroll2003,Carroll2006}. Derived from Activity Theory \cite{Leontev1974},
activity awareness encompasses information covering all aspects of collaboration, such as partner presence, mediating artifacts, group actions, social interactions, shared information, and group values and norms. Activity awareness has been utilized as a design concept in guiding and evaluating collaboration features.

% place system figure here simply for paper layout
\begin{figure*}
\centering
\includegraphics[width=2\columnwidth]{./img/interface.png}
\caption{CAnalytics user interface. Each window is closable, movable, and resizable. Shown here are \emph{document} view (top left), timeline (bottom left), network (top right), map (bottom middle), and message (bottom right). Other windows include table, notepad, and history view. }\label{fig:canalytics}
\end{figure*}

Many lab studies have been reported to investigate specific awareness features
to support collaboration in intelligence analysis. For example, Convertino et al. \cite{Convertino2011} examined the use of public and private views for role based collaboration. Goyal and
Fussell \cite{Goyal2016} studied the effect of hypotheses sharing on
sensemaking. Mahyar and Tory \cite{Mahyar2013b} designed a
visualization to connect collaborators' common findings and evaluated
its support for team performance. Hajizadeh et al.
\cite{Hajizadeh2013} explored how sharing teammate's interactions
affects awareness. These studies provide evidence to validate hypotheses of specific design
features. However, due to time constraint (mostly within one hour), these studies had to employ
a simplified task with reduced content and complexity. Artifacts created by participants were thus relatively simple and superficial
(e.g.~with a single artifact or few items in an artifact). More complex
task would have pushed participants to create more sophisticated artifacts
(e.g.~multiple views or cluttered display that requires filtering) and
to try balancing between team coordination and individual analysis, which
would have provided more insights into team-based analytic process. Our study aims to examine tool usage in a course project, which allows for higher task complexity and longer teamwork time, thus gaining more insight into the team process.
