
\section{Related work}\label{related-work}

Many studies have been reported to investigate specific design features
to support collaborative information analysis. For example, Goyal and
Fussell \autocite{Goyal2016} studied the effect of hypotheses sharing on
sensemaking. Mahyar and Tory \autocite{Mahyar2013b} designed a
visualization to connect collaborators' common findings and evaluated
its support for team performance. Hajizadeh et al.
\autocite{Hajizadeh2013} explored how sharing teammate's interactions
affects awareness. These studies report interesting results of
controlled lab studies to validate hypotheses of specific design
features. However, they do not provide insight on how teams would
collaborate with a complete tool as nexus of features in the real world
over extended period of time.

Field studies were conducted aimed to understand design requirements of
collaborative information analysis in more realistic settings. Chin et
al. \autocite{Chin2009} observed and analyzed the analytic strategies,
work practices, tools ad collaboration norms of professional
intelligence analysts. Kang and Stasko \autocite{Kang2011} studied how
student analysts, as in our study, completed in-class intelligence
projects. Carroll et al. \autocite{Carroll2013} attempted to model a
complex analytic task scenario in a lab setting, and examined the
development of team awareness in a four-hour-long task. These studies
helped improve understanding of current work practice with
state-of-the-art tools or no tools at all. We built our tool based upon
their study findings, and pursue to further explore design implications
by investigating tool usage in a similar naturalist environment.

Our study took place during the 10th week of the course. Before that
students learned several analytic techniques, including IEW (a technique
to extract and assess values of evidence), ACH (a technique to evaluate
multiple hypotheses against evidence), timeline analysis and network
analysis, as well as state-of-the-art tools to implement these
techniques. In the eighth weeks students practiced applying these
techniques in a hands-on project. A typical workflow started with IEW to
extract and model evidence from documents, followed by building analytic
artifacts such as an ACH Matrix in PARC ACH, a timeline and a network
graph in Analyst's Notebook. Since data from IEW table could not be
shared or extended to other tools, students had to replicate data for
each different tool. Most tools they used lacked serious collaboration
support (except that some teams used Google Doc to construct an IEW
table). Analysts were unable to contribute simultaneously (an issue
known as production blocking \autocite{Diehl1987a}). The analysis work
was often divided by tools: each person created and analyzed an artifact
with a tool on their own. This had the consequence that findings and
hypotheses be made without integrating collective efforts and diverse
knowledge. Analysts must coordinate work by manually sharing notebooks
or graphs through email or cloud storage service (e.g.~Dropbox),
resulting in a scattered placement of results, requiring repeated manual
resynchronizing to identify redundant or missing pieces of information,
analysis of information, and analytic hypotheses. The instructor and
students in our study were all aware of the shortcomings of available
tools with respect to support of collaboration.
