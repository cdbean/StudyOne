\subsection{User awareness}
% user awareness

% survey result
The awareness support provided by CAnalytics was received well. In the survey, we asked students to rate how aware they were of teammates' past activities, current activities, and future activities, and got 90\%, 76\%, and 64\% of positive ratings respectively. 
% MAYBE we just need to report overall awareness? 
% 88\% of students rated positively on their team awareness. 
% qua result
In the response to the question \textit{how do you think the tool help you stay aware of your teammates' activities}, we found several types of awareness students reported:

\textit{Information awareness}. % what information do I have?
% TODO description of information awareness
CAnalytics updated user-generated data objects in real time, and pushes incidental notification to teammates. 
\textit{`CAnalyitics was very helpful in keeping us updated on what was being changed/noted/amended by whom and when. This was very beneficial for staying on the same page and knowing what changes were being made so no one individual was out of the loop'}. Keeping shared information up to date also prevents redundant work. 

\textit{Social awareness}.  % who is contributing?
% TODO description of information awareness 
The user icon in the right top corner indicates who is online. The dot in the tool menu also gives users a peripheral view of who is activity working. 
\textit{`During class I wasn't sure if my teammates were doing work for that class or another thing but then seeing their dot switch between applications on the software and updates pop up on my screen I knew they were doing work'}
One students also reported that such awareness kept him `peace at mind' as he knew the team was working efficiently on the project. 
% TODO social identify may need further explanation

Besides, since the tool records activities individuals perform, students are able to see how much each person has contributed, thus forming a social identify.

\textit{Action awareness}.  % who is doing what?
% TODO description of action awareness
Similar to radar view \cite{Gutwin1996}, the user dot indicates where an individual is working on. But radar view does not apply to flexible interface, whose spatial layout tends to flow to user needs. Students were able to use the little dot on the tool menu to determine what teammates were doing. For example, when a dot is displayed in the document tab, it indicates that a teammate might be working on annotating evidence; and if they are in the node-link graph tab, they might be drawing inferences from the collected evidence.

Awareness of what teammates are doing avoids conflict. As one commented, 
\textit{`If *** [teammate name] was cleaning up the network map, I knew that working on the same thing might make more clutter at the time'}

Awareness assures the team that everyone is following the team strategy and is doing as expected. For example, if the team adopts a divide-and-conquer strategy, knowing where each member is working on and getting real time update of changes they made assures that they are doing the right thing.

\textit{History awareness}. % who has done what
% TODO description of history awareness
The history tool was not used extensively. 
% TODO: number of history tool that has been used. 
It is expected, however, as history tool only provides a meta-information about group activities, it is not expected to be used as a major tool for information analysis. Of the participants who used it, they commented the tool positive. They found the team history useful to learn what the team had done, especially when in asynchronous collaboration. 
\textit{`I also liked the history so if either I wasn't on or maybe just missed something, I could just pull that up and see what has been done.'}
One student also mentioned that the history view helps him keep track of group direction, so that he knows what to do next. 
\textit{`The history function allowed me to see what my teammates changed so I could continue to contribute in a team direction.'}
Looking back in the history reminds individuals of the bigger picture, and the progress the group has achieved. 
\textit{`you can check all of the changes that were made and when so that my teammates and I could check the progress of the project and see who exactly was making changes to what.'}


% consequence of awareness

\vspace{5mm} % vertical space

Achieving sufficient awareness of teammates facilitates information sharing. Convertino \cite{Convertino2011} has found evidence that individuals tend to  more frequently  offer to ``push'' information before being asked as they know each other better. One comment from a student echoed that finding.
\textit{`If I were to come across a piece of information that I thought might be helpful to them I would just tell them. My teammates did the same thing in return.'}
Such push action is in turn continuously reminded by the switching of user dots and updated notification of user activities. For example, one student shared his experience: once he saw his teammate was on the network tool, he knew he was attempting to connect the information. He then purposely kept an eye on related findings that might influence his work when he was working on his own track.

Such awareness also makes collective efforts easier. This is reflected in two aspects. When people make a mistake, teammates are able to identify that mistake and correct it. 
% TODO mistake correction
Since the team shared the same data objects and visual artifacts, they were able to build upon prior work and avoid repeating. 
\textit{`they [teammates] were able to work directly off something that I had also created. This ability to work off of each other’s own work allowed us to all contribute to each element of the analytic process'}. For example, one student went through one of the documents and created a base node-link graph that included all information he considered relevant. His teammate could later include information in the chart that was missed. 
\textit{`He was able to work off of my initial diagram and add and manipulate it in order to include additional relevant information I had missed.'}



% awareness breakdown
%% strategy awareness
%%% how to annotate
Apart from positive awareness support, we also asked breakdowns students had experienced. Students reported that they had problem coordinating team strategies. For example, although they were aware of \textit{what} annotations teammates had created, they seemed to have problem coordinating \emph{how} to annotate. For example, some groups did not establish a consistent annotation rule. When some members annotated banks as organizations, others marked them as locations. When naming a data object, one member might name a robbery event using the bank where the robbery occurred, such as ``National Bank robbery'', whereas another member might use the date when the robbery happened, such as ``January 12 robbery''. Both approaches made sense, but being inconsistent resulted in confusion. 
%%% what granuality of information to annotate
Another issue was to coordinate the level of information the team was to gather. For example, some members started with collecting only high-level information, such as related people and locations, while others might annotate detailed they believed important, such as the robber's action sequence, in an attempt to examine if common behavior patterns exist between different cases. For instance, one group created 223 entities, way more than the average (=76). One group member annotated every single suspect's action as an event when the other members were offline. As one member reflected,
\textit{`We had problems coordinating what we should annotate. We didn't really have direction and were going about it in different ways...'}
However, depending on the short-term group goal, detailed information might not help, e.g. when the group only started to try to have a big picture of the whole case. Worse, these annotated objects were automatically visualized, causing the shared artifact unnecessarily complex. 
\textit{`I felt it was very easy for everyone to get carried away highlighting things which would get out of control and we would have information that was not very important to solving the problem highlighted'.}

%% visualization sharing
In the current system, although user created objects are automatically shared, the view of those objects are not shared; that is, further user manipulation to the visualization such as zoom, filter, highlight, drag that reorganizes the view is for individuals only. Views were kept private to avoid interference. However, students requested to have a way to share their visualization state as well. Having different views could lead to confusion because they \textit{``could be looking at the same information but arranged in completely different ways''}

%% information and interpretation of information
Students also mentioned the contrast between \emph{sharing of information} and \emph{sharing of interpretation of information}. Although the information they collected was immediately shared, individuals could have different or conflicting interpretations toward the same information. Yet the students also acknowledged that \textit{"this problem occurred within any group project, not just when using CAnalytics"}