\section{Introduction}\label{introduction}

Intelligence analysis is a form of sensemaking wherein a team of analysts
identify and evaluate causal relationships in a complex corpus of documents. A
common example of such an analysis is the investigation of a crime, where a variety of putative facts are assembled, e.g.
financial records, witness observations and interviews, and the social
connections between persons of interest. Armed with these different assembled facts
 investigators collaboratively: assess means, motives,
and opportunities; articulate and investigate further hypotheses and deductions;
and develop one or more theories of the crime.

%\bvh{Outline the critical challenge of modeling data and basing hypotheses on
% it. Current tooling forces you into waterfall.} \bvh{I would say the waterfall
% point first, and be more explicit about how it forces you into that model.}

According to Pirolli and Card's model \cite{Pirolli2005}, intelligence analysis
starts with modeling data from textual documents, followed by representing these
data models in various information artifacts, and developing them into
hypotheses.  Systems that currently support intelligence analysis are aimed at a
single phase and therefore only support part of the overall analysis workflow.
This imposes a clear boundary between each of these phases on the analysts. For
example, the Analysis of Competing Hypotheses (ACH) tool assumes that data has
been modeled, and that relevant evidence can be adduced appropriately to various
hypotheses, but provides no structured support for either. Other techniques,
such as Information Extraction and Weighting (IEW), help structure evidence
modeling, but do not extend utilization of evidence to hypothesis generation.
The unintended boundary between phases has the consequence that data modeled in
one software cannot be effectively utilized in hypothesis development in another
system. And analysts have to handoff, often via replicating the data in the new
system, information between software systems, making it difficult to revisit and
revise the data model. We therefore are motivated to develop an integrated
workspace in which analysts can model and analyze data in one place, and we
utilize the system as an experimental instrument to investigate analytic
behavior afforded by such integration.

% \bvh{Talk about how it is collaborative.} \bvh{Although I get lost in what you
% are trying to do in the end of this paragraph...I am not sure what point you are
% trying to make. This is the most confusing part of the intro so far}
Any intelligence analysis activity, at least at a non-trivial scale, is
fundamentally collaborative \cite{Convertino2011}. The intelligence community
puts great value on collaboration. A report from the Director of National
Intelligence, \emph{Vision 2015},  called for \emph{``a dramatic shift from
traditional emphasis on self-reliance toward more collaborative operations''}
\cite[p.13]{Vision2015}. However, most analytic tools that are widely used in
the intelligence community (e.g.~Analyst's Notebook \cite{IBM} and PARC ACH
\cite{PARC}) do not support collaboration. The community has to rely on separate
collaboration tools (e.g.~email, Intellipedia \cite{Intelink2017}), which lack
serious support for analytics, for team coordination \cite{Treverton2016}. Thus,
analysts must coordinate their work outside of their tool support, manually
sharing their analytic products.

Supporting collaboration in intelligence analysis is challenging, and perhaps
more than other situations (e.g.~collaborative writing, wiki) because the task
itself can be extremely complex. A team could be working with much more complex
data structure (e.g.~spatial data, temporal data, and relational data, as
opposed to text only), coordinating through multiple analytic artifacts
(e.g.~visualizations as opposed to document only), and making
sense of different levels of analysis, assumptions, and hypotheses, both
synchronously and asynchronously throughout a long-term course of collaborative
interaction. Thus teams must not only stay aware of what other members are doing, but \emph{why} they are doing that in a specific context of analysis. This study aims to investigate how technology can mediate team collaboration and to understand what awareness is needed beyond team actions.

%\bvh{Talk about what we did, this makes sense.}
We performed our study within the context of an intelligence analysis course.
This classroom study provides a natural environment in which participants engage
in multi-session, relatively complex class projects. Due to the difficulty in
accessing professional analysts due to security and confidentiality issues,
studying \emph{analysts-in-training} provides us a chance to include them in a
longitudinal design loop. These students already have knowledge and experience
with state-of-the-art analytic techniques and tools and are thus more likely to
provide insightful feedback. Besides, the students are young learners that are
willing to employ new work practices supported by features in tools. They are
important parts of the future intelligence community. In some sense, their
practice can be treated as a view into the future of practice of the community
\cite{Martin2014}.

We are thus motivated to investigate the feasibility, effectiveness and
consequence of supporting collaborative intelligence analysis in the context of
classroom study. We have developed a tool that includes annotation for data
modeling, interactive visualization for data analysis, and collaboration
features. While this paper reports a collaborative task in a specific domain,
findings regarding team process and breakdowns meet the interest of the broader
CSCW community. This study makes three contributions: 1) we observed a
spontaneously adopted interleaving workflow and quantitatively proved that an
earlier switch from modeling to analysis improved performance; 2) we
distinguished three collaboration strategies, five factors that impacted
performance, and issues that caused team breakdown; and 3) we explored further
the design of awareness support and proposed awareness surpasses simply teams
actions but includes contribution value, uncertainty, and context of insight.
