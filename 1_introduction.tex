\section{Introduction}\label{introduction}

Collaborative information analysis is a form of sensemaking wherein a
team analyzes a complex information space of facts and relationships to
identify and evaluate causal hypotheses. A common example of
collaborative information analysis is crime investigation; a variety of
putative facts are assembled, including financial records, witness
observations and interviews, and social connections of various sorts
among persons of interest, from which investigators collaboratively
assess means, motives, and opportunities, articulate and investigate
further hypotheses and deductions, and develop one or more theories of
the crime. Other examples include intelligence analysis, business
intelligence, scientific research, and social constructivist learning.

A critical challenge for information analysts is building an adequate
preliminary data model from textual documents, and insuring that the data model
is employed effectively in hypothesis development and evaluation. This is an
open challenge \autocite{Badalamente2005}. Standard methods often do not support
it at all; for example, Analysis of Competing Hypotheses (ACH) assumes that data
has been modeled, and that relevant evidence can be adduced appropriately to
various hypotheses, but provides no structured support for either. Visual
analytics systems, as Ware termed as ``asymmetry in data rates''
\autocite[382]{Ware2012}, emphasized data flowing from visualization systems to
users far more than from users to systems. Functionalities are mostly designed
to adjust visual representation rather than remodel data underlying the
representation, which is a critical aspect in information analysis. In contrast,
other techniques, such as Information Extraction and Weighting (IEW), help
structure modeling of evidence, but do not add analytic support or extend
utilization of evidence to hypothesis generation. We therefore are motivated to
develop an integrated workspace in which analysts can model and analyze data in
one place, and investigate how that will affect analytic process.

Any work of information analysis at a non-trivial scale is fundamentally
collaborative. A key enabler for effective collaboration is
\emph{activity awareness}, defined as team's awareness of its own
sustained collaborative activity \autocite{Carroll2006}. Derived from
Activity Theory, activity awareness transcends synchronous awareness of
who collaborators are, where a collaborator is looking, etc. It
encompasses issues of many different kinds of information covering all
aspects of an activity, such as events, tasks, goals, mediating
artifacts, social interactions, and group values and norms, which
becomes higher demanding as the activity becomes more complicated.
Awareness support in such a complex activity of information analysis is
perhaps also more challenging than many other situations
(e.g.~collaborative writing). Teammates could be working with much more
complex data structure (e.g.~spatial data, temporal data, and relational
data, as opposed to only text), coordinating through multiple analytic
artifacts (e.g.~map, timeline, network, as opposed to only a document),
and making sense of different levels of analysis, assumptions, and
hypotheses, both synchronously and asynchronously throughout a long-term
course of collaborative interaction. Hence we will investigate how
technology can mediate team collaboration in a complex analytic task
over extended period of time.

We situate our study in classroom learning of information analysis.
Classroom study provides a natural environment in which participants
engage in long term, complex class projects. Due to difficulty in
accessing professional analysts or having them in long term design
loops, analysts in training who are learning to be information analysts
are a good compromise \autocite{Kang2011}. They already have some
knowledge and experience with state-of-the-art analytic techniques and
tools. In class projects students are graded on their ability to
understand and enact professional practices of information analysis.
This strong normative emphasis on problem solving practices is a great
evaluation context for new interactive tools: Tools are only valuable to
the students insofar as they actually support better practices and
better outcomes.

We thus are motivated to investigate the feasibility, effectiveness and
consequence of supporting integrated data modeling and analysis, as well
as supporting activity awareness in complex information analysis, in the
context of classroom study. We have developed a tool that includes
annotation for data modeling, interactive visualization for data
analysis, and a set of awareness features. For the balance of this
paper, we describe the tool we have developed, the classroom settings,
and our observations. We conclude with design implications derived from
the study as well as future work.
