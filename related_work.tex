% related work
\section{Related Work}
\hl{not modified yet}

% need for collaborative information analysis
The i2's Analyst's Notebook and and PARC ACH \footnote{Analyst's Notebook: http://www-03.ibm.com/software/products/en/analysts-notebook, ACH: http://www2.parc.com/istl/projects/ach/ach.html}are widely used tools for information analysis, and emphasize support for visualizing events in temporal themes and for competitive analysis of alternative hypotheses, respectively. Recent research has focused on visualizing entity and relationships both more directly, from the user's standpoint, and through multiple view techniques. Entity-based systems such as the Entity Workspace \cite{Bier2010} and Jigsaw \cite{Jigsaw2008} extracts data objects including people, location, and times, and represent them in organized graphical structures. All these systems attempt to reduce the cost of manipulating text into connected data objects. 

However, these tools lack serious collaboration support. Analysts must coordinate work by manually sharing notebooks or graphs. This has the consequence that the tools are employed only at specific points in an analysis, often only in the early stages during which analysts are working on their own. Indeed, the use of these tools directly diminishes a team's awareness, requiring repeated manual resynchronizing to identify redundant or missing pieces of information, analysis of information, and analytic hypotheses. The instructor and students in our study were also aware of the shortcomings of available tools with respect to support of collaboration.

Carroll and his colleagues \cite{Borge2012, Carroll2013} conducted a lab experiment and observed how teams collaboratively analyzed information using paper prototype. By analyzing team process and artifacts teams developed, they derived design suggestions for mediating tools.  Goyal et al. \cite{Goyal2013, Goyal2014, Goyal2016}  reported several lab experiments investigating the effect of individual tool features on collaborative information analysis. Their work provides evidence of tool effect on collaboration. Other studies have been conducted to examine the technology to support group awareness. 
% TODO include citations 
These technologies include visualizations of working areas of collaborators \cite{Gutwin1996}, highlighting of changes \cite{Hajizadeh2013}, notification systems \cite{Carroll2003}, and historical activity list \cite{Ganoe2003}. These studies were all conducted in lab experiments, which focus on relatively simple tasks and short periods of time. As task complexity increases and time expands, however, team behavior could be qualitatively different. % TODO citation

% BVH Changed this from using passive voice
Researchers have conducted several field studies aimed to understand design requirements of collaborative information analysis in more realistic setting. Chin et al. \cite{Chin2009} conducted an observation study of professional intelligence analysts, and examined the strategies and tools they applied. Their study is important as it sheds light on how real professionals work. In practice, access to professional analysts over an extended period of time is often restricted due to safety and privacy issues. As an alternative, Kang and Stasko \cite{Kang2011} studied how student analysts---who were soon to become professionals---completed in-class intelligence projects using state-of-the-art tools. This approach proves to be a practical way to understand and involve users in the problem space. Our study followed this approach, and took one step further; we deployed our tool in an intelligence analysis course and took advantage of the course as a large-scale testbed to investigate the tool we have developed. 


% system



%i2's Analyst's Notebook
%
%Stasko, J., Görg, C., & Spence, R. (2008). Jigsaw: supporting investigative analysis through interactive visualization. Information Visualization, 7(2), 118–132. http://doi.org/10.1057/palgrave.ivs.9500180
%
%Bier, E. a., Card, S. K., & Bodnar, J. W. (2008). Entity-based collaboration tools for intelligence analysis. In 2008 IEEE Symposium on Visual Analytics Science and Technology (pp. 99–106). Ieee. http://doi.org/10.1109/VAST.2008.4677362
%
%
%% empirical study, including lab experiment and longitudinal field study, and evaluation method
%
%professionals are difficult to access for longitudinal observation
%
%% awareness issue
%
%% evaluation
%
%Carroll, J. M., Borge, M., & Shih, S. (2013). Cognitive artifacts as a window on design. Journal of Visual Languages & Computing, 24(4), 248–261. Retrieved from http://www.sciencedirect.com/science/article/pii/S1045926X13000207
%
%Borge, M., Ganoe, C. H., Shih, S.-I., & Carroll, J. M. (2012). Patterns of team processes and breakdowns in information analysis tasks. In Proceedings of the ACM 2012 conference on Computer Supported Cooperative Work - CSCW ’12 (pp. 1105–1114). New York, New York, USA: ACM Press. http://doi.org/10.1145/2145204.2145369
%
%Goyal, N., Leshed, G., Cosley, D., & Fussell, S. R. (2014). Effects of Implicit Sharing in Collaborative Analysis. In Proceedings of the Proceedings of the SIGCHI Conference on Human Factors in Computing Systems (pp. 129–138). Retrieved from http://www.cs.cornell.edu/~ngoyal/1470_Chi2014.pdf
%Goyal, N., & Fussell, S. R. (2016). Effects of Sensemaking Translucence on Distributed Collaborative Analysis. In CSCW ’16 Proceedings of the 18th ACM Conference on Computer Supported Cooperative Work & Social Computing.
%Goyal, N., Leshed, G., & Fussell, S. (2013). Effects of visualization and note-taking on sensemaking and analysis. In CHI ’13 Proceedings of the SIGCHI Conference on Human Factors in Computing Systems (pp. 2721–2724). Retrieved from http://dl.acm.org/citation.cfm?id=2481376
%
%Chin Jr, G., Kuchar, O. A., & Wolf, K. E. (2009). Exploring the analytical processes of intelligence analysts. In Proceedings of the SIGCHI Conference on Human Factors in Computing Systems (pp. 11–20). Retrieved from http://dl.acm.org/citation.cfm?id=1518704
%
%Kang, Y., & Stasko, J. (2011). Characterizing the intelligence analysis process: Informing visual analytics design through a longitudinal field study. In 2011 IEEE Conference on Visual Analytics Science and Technology (VAST) (pp. 21–30). Ieee. http://doi.org/10.1109/VAST.2011.6102438
