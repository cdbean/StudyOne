\section{Discussion}\label{discussion}

The goal of the study is to understand and support an interleaved workflow in
collaborative information analysis by evaluating tool usage in a natural
environment over multiple usage sessions. Our work responds to calls for an
integrated environment made in the intelligence community as well as other data
intensive domains \cite{Shah2014i, Chen2016, Vision2015, Amershi2015, Ware2012}.
The system builds upon prior empirical studies (e.g. \cite{Carroll2013,
Borge2012,Kang2011,Chin2009}) and embodies their design implications in our
tool. The study also complements research that only tests tools in short term
lab studies (e.g. \cite{Convertino2011,Goyal2016,Hajizadeh2013}) by
investigating tool usage over multiple usage sessions.

This study adopts an \emph{analyst-centered design} approach. A critical
requirement of developing tools that meet user needs is to understand their
needs and practices. When these needs and practices are specialized (as is the
case in intelligence analysis), it is particularly important to include the
target user population in the design process \cite{Scholtz2014}. Our classroom
study provided an opportunity with deep access to analysts in training. These
analysts have been trained with knowledge and skills of intelligence analysis,
and have experience with state-of-the-art analytic tools such as Analyst's
Notebook and PARC ACH. In their reflections, participants often compared
CAnalytics to those tools. Their multi-session usage of CAnalytics also allows
them to adapt to the tool and learn to appropriate it to the best of their team
purpose \cite{Stahl2006}. Therefore, while their feedback is admittedly not the
same as an experienced professional, their feedback does provide a deeper
insight into strengths and weaknesses of CAnalytics.

The study provides encouraging results. Participants appreciated an all-in-one
environment where they could share raw documents, evidence snippets, views of
evidence and hypotheses in one place. They liked the fact that they could
contribute simultaneously without blocking or interfering each other. Another
benefit of the collaborative tool is to keep teammates aware of each other's
activities. Staying aware of teammates not only helps establish a common ground
for planning team strategy, but also ensure everyone is following the plan as
expected. Moreover, results suggest the awareness features provide positive
\emph{social facilitation} \cite{Zajonc1965}: individuals found the task
motivating and engaging with awareness of each other's activity. We also
measured collaboration characteristics that impacted team performance, and found
that balanced contribution, peer editing, and earlier switching from modeling to
analysis predicted higher team performance. Most importantly, we documented the
interleaved workflow enabled by the integrated environment, and explored
momentum in modeling and analysis behaviors that drove the activity switching.
Below we discuss design implications that could potentially enable a better
collaborative experience in information analysis tasks.

\paragraph{Scaffold a structured interleaved workflow}

A misconception about information analysis is that data modeling and
data analysis are two staged activities. This is akin to the \emph{waterfall}
software development model, which features a sequential process that
flows downwards through the phases of requirement conception, software
design and implementation, and testing and maintenance. Critics have
argued against this approach and instead called for an iterative design
process that leads to reframe user requirements,
redesign, redevelopment, and retesting.

Our result demonstrates a similar iterative and dynamic process in intelligence
analysis. The result is striking especially because our participants have been
trained with tools that impose a waterfall model. They could have followed their
old waterfall practice with our tool, yet instead all teams spontaneously
switched to an iterative manner. We projected that modeling on multiple granularities drives analysis on different levels, and uncertainty in analysis pushes analysts back to collecting additional data.

Realizing that, we probably could shape analysts into a more interleaved
workflow with a more structured approach. Structured techniques help users
perform analysis in a systematic and transparent manner. For example, IEW aims
to structure evidence modeling, and ACH proceduralizes evaluation of analytic
hypotheses. Yet the way to switch between the two activities is loosely defined
\cite{Kang2011}. Our system implies a structured modeling through annotation and
a structured analysis by visualizing data in multiple views. By sharing the same
data structure and consistent user interface, we enable a smooth switching
between the two stages. Our result implies the role of multilevel modeling and
analysis uncertainty in driving the switch. Based on that, there exist design
opportunities to enable a more interleaved flow. For example, we can build a more structured scaffolding process to bridge modeling and analysis. When user adds a
new data object, the system could suggest possible connections to existing
evidence in the context of an appropriate level of views, which is likely to
help analysts discover new patterns. When a user creates a new hypothesis with
uncertainty, the system could highlight associated evidence, which would prompt
the analyst to re-examine the data and look for more data. Such scaffolding
provides a basic structure to link stages of analytic activities that analysts
can take on without imposing a specific fixed workflow.

A smoothier interleaved workflow could also be made by increasing team
awareness of partner's modeling and analysis. That is, uncertainty in one's
analysis not only motivates oneself to data modeling, but also drives partners
(who is aware of the what and why of the uncertainty) to collect more data.
Similarly, one's modeling could influence and drive partner's analysis, given
the partner is fully aware of what is modeled and how the new data connects to
the level of existing data. Such ``team-level'' interleaving could make teamwork
more interactive and close coupling, but also requires support of higher
awareness, especially of hypothesis uncertainty and data model context. Improved
design for multi-granularity modeling, uncertainty representation, as well as
team awareness of these features opens up possibilities for coordinating
interleaving at the team level. We discuss these design implications in further
detail in the following paragraphs.

\paragraph{Build collapsible views for multi-level modeling}

We observed that analysts built data models in multiple granularities and engaged and coordinated among different levels of details. For example, analysts jumped
quickly from digging into details of a single event, to comparing between two
events, and to overviewing all robberies as a complete story. When sharing these data, a critical requirement is to represent them in an appropriate context in order to ensure teammates understand them correctly. A collapsible data view could be a solution to accommodate such
multi-level modeling. This can help draw teammate's attention to a specific item
while keeping the global context available. Analysts can focus on a certain
level of detail at a time while conveniently switching between levels. A
collapsible view also reduces the problem of cluttered view when data volume
increases.  and when analysts dig into greater
details (e.g.~representing suspect's all actions to identify patterns of common
actions in two robberies). An analyst can overview all robberies and only unfold
detailed actions when investigating into a specific robbery.

\paragraph{Design a richer graphic language for uncertainty}

Uncertainty is prevalent in analysis and more prevalent in collaborative
analysis because each teammate becomes a source of uncertainty when they
contribute a piece of information \cite{Chin2009}. Teams in our study
spontaneously employed two different approaches to deal with uncertainty: either
to mix them for better synthesis or to separate them for better clarification.
Representing uncertainty appropriately assures teammates when they build
analysis upon other's insight. This helps reduce the problem emphasized in Chin
et al.'s work \cite{Chin2009} that analysts do not trust partner's conclusion.
It also helps teammates stay aware of issues that need to be addressed, thus
decide what extra data is needed. We propose that a richer graphic language and
interaction be designed so that analysts can encode uncertainty into the views.
For example, links and entities with different uncertainty can be visualized in
different transparency. Users can \emph{filter} by uncertainty so that users can
choose to consider evidence only with high credibility, or to review all
inferences and decide what extra data is needed to consolidate them.

\paragraph{Share views as team resource}

View sharing is important for sharing and understanding analytic result
\cite{Morton2014a}. A common solution is to enable a public view which always
keeps synchronized for all teammates \cite{Convertino2011,Greenberg1990}.
However, a single public view does not meet the need in intelligence analysis
because analysts may want to share multiple views in an iterative analysis
process. We propose that views should be treated as a \emph{team resource}, just
like data, which is sharable, extensible, and reusable. For example, an analyst
can save their current view as a shared resource when they feel it useful to
collaborators. Other people can reuse the view to their need. Shared views
should be interactive rather than static images, so that analysts can perform
all interactions including filtering and highlighting, and are able to evolve
the view with collective team efforts, a critical requirement emphasized in
\cite{Carroll2013}.

\paragraph{Distinguish visible vs. valuable contributions}

Finally, we noted cases where teams created far more entities than needed with an
accretion strategy. Strikingly, while similar data modeling strategy was
reported in the paper prototype study \cite{Carroll2013}, users with CAnalytics
seemed far more tempted to accretively add information, with far more entities
and cluttered views. For example, the extreme team created as many as three
times entities than the rest teams in our study, much more than the difference
in the paper prototype study. Why did this happen? We guess both the context of
classroom study and the system design contributed. Unlike in the lab study where
teams are temporarily assembled, teams in a class evaluate peers either
consciously or unconsciously and value how themselves are being evaluated. Such
social pressure motivate individuals to make contributions, and indeed to make
\emph{visible} contributions, more than \emph{valuable} contributions. That is,
participants noticed that their work activity was visible to their partners, and
accordingly prioritized doing more visible work over doing less visible work. In
some cases, this led to a new problem of easy and less valuable contributions
that were highly visible - such as creating and therefore sharing data entities
that were not particularly important, and subsequently made data models seem
cluttered. For example, creating and therefore sharing an entity gets
immediately notified to the team whereas weighing the importance and relevance
of an entity goes silent in the system. We need to investigate approaches to
making significant contributions more visible, or perhaps making it more
immediately visible that less important contributions are indeed less important.
