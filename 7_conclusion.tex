\section{Conclusion}

In this paper, we present findings from a classroom study in which teams of intelligence analysts in training collaboratively completed a complex intelligence project mediated by our tool. Our study provides encouraging results on supporting collaborative analysis. Analysts spontaneously employed an interleaving workflow and making use of awareness features provided by the tool to stay team awareness. We also identify new opportunities in awareness support; awareness includes not only team actions, but information uncertainty, contribution visibility, and context for insight sharing. We propose design implications to enable a more interleaving workflow and possibly at a team level.

While this study was performed with an emphasis on intelligence analysis, we argue that CAnalytics can also be utilized for a broader domain of collaborative information analysis, wherein a team examines a complex information space of facts and relationships. These examples include business intelligence, scientific research, and social constructivist learning, in addition to intelligence analysis. As collaborative information analysis is increasingly a typical and chronic task, it is important for research to examine, understand, and provide effective tools and environments for these long-term, real-world CSCW interactions. This requires situating research in more complex work activity contexts, and directly investigating interactions, experiences, and outcomes in those contexts.

Albeit the advantages of classroom study, this study comes with constraints and limitations. For example, many factors and
variables could exist that affect team performance. The fact that these
factors are often impossible to model or control adds to the difficulty
in data analysis (as mentioned beforehand, we were unable to conduct a control experiment). Also, data collection is
challenging because team interactions are not always accessible. Teams
can choose to work synchronously or asynchronously, and it is difficult
to predict when or where the interaction of most interest is to occur.
Verbal communication is not accessible, which could be useful to infer
team awareness as a complement to interaction logs. Lab studies
and case studies can be conducted in the future to address these
problems with greater control and deeper data access.


% Meanwihle, due to the nature of classroom study, a lot of variables that cannot be factored could influence the result. For example, students coordinated teamwork not only through CAnalytics but also through face-to-face meeting. What the teams did outside CAnalytics is not collected, which obviously influenced team performance. But our work provides a realistic picture of usage of a collaborative tool in real world, and poses several design questions and hypotheses, which could be investigated in future controlled lab studies.

%Since the study was conducted in a naturalist environment, participants were not observed in a controlled lab, and they were free to choose either face-to-face collaboration settings or remote work settings, and they could also use any other tools they preferred. Thus their high rating of awareness in the survey is not a sufficient proof of the success of the tool. Still, their answer in the open-ended questionnaire explained to some extent how the tool helped achieve the function.
%
%
%establish the dots and connect the dots
%
%
%
%This paper talks about strength and weakness of lab and field study, and suggests combining the two methods.
%A laboratory method for studying activity awareness. Convertino. 2004
