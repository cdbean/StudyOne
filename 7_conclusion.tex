\section{Conclusion}

In this paper, we present findings from a classroom study in which teams of information analysts in training collaboratively completed a complex intelligence project mediated by our tool. As collaborative information analysis is increasingly a typical and chronic task, it is important for research to examine, understand, and provide effective tools and environments for these long-term, real-world CSCW interactions. This requires situating research in more complex work activity contexts, and directly investigating interactions, experiences, and outcomes in those contexts. Our classroom study provides initial results on team interactions mediated by advanced technology over extended time periods. The encouraging results motivate us to continue refining and re-evaluating the tool.

% Meanwihle, due to the nature of classroom study, a lot of variables that cannot be factored could influence the result. For example, students coordinated teamwork not only through CAnalytics but also through face-to-face meeting. What the teams did outside CAnalytics is not collected, which obviously influenced team performance. But our work provides a realistic picture of usage of a collaborative tool in real world, and poses several design questions and hypotheses, which could be investigated in future controlled lab studies. 

%Since the study was conducted in a naturalist environment, participants were not observed in a controlled lab, and they were free to choose either face-to-face collaboration settings or remote work settings, and they could also use any other tools they preferred. Thus their high rating of awareness in the survey is not a sufficient proof of the success of the tool. Still, their answer in the open-ended questionnaire explained to some extent how the tool helped achieve the function.  
%
%
%establish the dots and connect the dots
%
%
%
%This paper talks about strength and weakness of lab and field study, and suggests combining the two methods. 
%A laboratory method for studying activity awareness. Convertino. 2004