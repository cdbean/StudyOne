% introduction
\section{Introduction}
\hl{not modified yet}

Collaborative information analysis is a form of sensemaking wherein a team analyzes a complex information space of facts and relationships to identify and evaluate causal hypotheses. A common example of collaborative information analysis is crime investigation; a variety of putative facts are assembled, including financial records, witness observations and interviews, and social connections of various sorts among persons of interest, from which investigators collaboratively  assess means, motives, and opportunities, articulate and investigate further hypotheses and deductions, and develop one or more theories of the crime. Other examples include intelligence analysis, business intelligence, scientific research, and social constructivist learning. Collaborative information analysis tools are designed to support such tasks.

However, Evaluation of collaborative tools is challenging for several reasons. For example, participants could address information in a qualitatively different manner. Participant's education, past experience, role requirement, cultural values, and organizational norms strongly influence the way they perceive information, what they perceive, and how they respond to this information \cite{heuer1999}. Therefore professional information analysts could have different strategies accomplishing a task and thus different perspectives toward the tool from casual users. Professional analysts have knowledge of practice of community such as structured analytic skills that is unavailable for untrained people. The problem is, however, professional information analysts such as intelligence analysts are often limited to access due to security issues. It is difficult to have them in a design loop and conduct a long term design research. 

Another challenge is to model the complex collaborative information analysis task as it is in reality. In a crime investigation situation, for example, teams interact with each other for weeks or months. During the period, analysts having multiple threads going on simultaneously. Issues that were raised an hour ago, or a week ago, may nonetheless be critical again. Collaborators need to keep track of how their teamwork is organized and managed, including team strategies and practices, and interdependent member responsibilities and roles. Analysts may be located in different places, and may have to work asynchronously together. All such complexity, while extremely common in real world, can hardly be modeled in lab study.



% Analysts seek to identify entities of interest from mass information and to determine relationships beyond those literally specified in the original data. A common example is crime investigation; a variety of putative facts are assembled, including financial records, witness observations and interviews, and social connections of various sorts among persons of interest, from which investigators collaboratively  assess means, motives, and opportunities, articulate and investigate further hypotheses and deductions, and develop one or more theories of the crime. Many critical problem domains beyond crime investigation involve or are examples of collaborative information analysis, including intelligence analysis, scientific research, and social constructivist learning.

% % BVH: mention supporting collaborative IA, raising awareness, and facilitating communication
% % DONG: modified the following paragraph to reflect the three elements.
% Collaboration is vital to information analysis. Prior studies \cite{Chin2009, Kang2011} have emphasized that the work of information analysis at non-trivial scales is fundamentally collaborative. However, the state-of-the-art analytic tools that are currently wide used, such as IBM Analyst's Notebook, and PARC ACH, are designed for individual use only. For successful collaboration, awareness of teammate's activities and effective communication is critical. Without effective support, analysts have to coordinate their work by manually sharing their work. The fact that the single person operating the software is observed to be a bottleneck to the team process \cite{Rosettex2004}. Tools to support and mediate such collaboration is critical; in particular, we need tools to support information analysis, to raise user awareness, and to facilitate communication.

% Besides, evaluation of collaborative tools is challenging, due to difficulty in data collection, number and complexity of variables, and the challenge to evaluate in real context \cite{Neale2004}. Real world information analysis often extends over long time periods involving synchronous and asynchronous team interactions, which can hardly be modeled in lab experiments. This requires situating research in more complex work activity contexts. On the other hand, however, professional information analysts in real world are often within limited access due to security and privacy issues. Attempts to observe their work and involve them in long-run iterative design is almost impossible. 

% To address these concerns and to support collaborative information analysis with complex tasks, we have developed a collaborative information analysis tool and deployed it in a naturalist environment. The design of the tool is informed by prior experiments of teams collaboratively analyzing crimes cases using paper prototype \cite{}, by field studies conducted by Chin \cite{Chin2009} and Kang and Stasko \cite{Kang2011}, and by existing systems \cite{Jigsaw2008, Bier2010}. The tool supports identifying analytic entities through document annotation and connecting and presenting information through multiple coordinated visualizations. We also built a set of features to increase user awareness, such as real time update of user created objects, an integrated notification system, and peripheral user working area indicators, and a dedicated user activity history.

In this paper, we reported a classroom study to evaluate a tool we developed for collaborative information analysis. We deployed our tool in an intelligence analysis course project in a US university. As an alternative to professional analysts, our participants are all analysts-in-training, who are learning state-of-the-art analytic skills and are soon to become professional intelligence analysts. They are familiar with the practice of the intelligence community and are instructed to follow that practice in their course project. The project task simulates real world intelligence tasks and requires teams to devote serious efforts to develop and valid hypotheses based on given materials. Students switch regularly back and forth between other course assignments and this project, and must accomplish the project under limited time and resources, similar to the real world situations of intelligence analysts. Students are graded in the course on their ability to understand and enact professional practices of information analysis. This strong normative emphasis on problem solving practices is a great evaluation context for new interactive tools: Tools are only valuable to the students insofar as they actually support better practices and better outcomes.

Particularly, we focus on the tool effect on three aspects: support for information analysis, support for activity awareness, and support for communication. Information analysis includes information gathering and synthesis, as well as hypothesis development and validation. Prior studies \cite{Chin2009,Kang2011,Borge2012} identified several requirements for effective information analysis, such as visual representation of data relationships, use of multiple variables to for data reduction, insight provenance for sanity checks, etc. Activity awareness \cite{Carroll2003} is a key enabler for effective collaboration. Collaborators need to keep track of what their partners have done, what they are doing, where they are working on, and their relevant experience, distinctive resources and unique knowledge, prior contributions and perspectives, as well as values, expectations, and goals. Finally, communication helps team coordinate and exchange opinions. Effective communication makes a team feel on the same page whereas ineffective communication wastes a lot of time in clarifying misunderstandings. We observe students' performance on these aspects and discuss the role of our tool in supporting or hindering them. 

Our work will be beneficial to design researchers and information analysis practitioners. Prior works have examined team behaviors in collaborative information analysis using state-of-the-art tools (current practice) \cite{Kang2011,Chin2009} or with low-fidelity tools (e.g. paper and pen) \cite{Borge2012}, or in a simplified laboratory task \cite{Goyal2016}. Our work is the first to observe team behaviors mediated by advanced technology in a naturalist environment, thus providing a more realistic picture of how tools are appropriated in team collaboration. 

A broader impact we anticipate is to contribute to technology-enhanced education of intelligence analysis. Methods in the intelligence community remain largely unchanged over the last few decades, and analysts are often trained with traditional methods. Our work is a starting point to bring advanced technology to intelligence training. It is important not only because it is likely to help students to learn and practice analytic skills in a more effective way, but more significantly, it opens up new possibilities for these future intelligence analysts to approach collaborative information analysis and encourage them to in turn push forward the development of technology in the area. 

% On the one hand, students employed our tool as an alternative to mediate their project activities; on the other hand, it served as a large-scale testbed for our tool. We report students' response to the tool and discuss design implications we find from the result.

% % TODO: cite Heuer to argue the importance of participants in information analysis
% ``What people perceive, how readily they perceive it, and how they process this information after receiving it are all strongly influenced by past experience, education, cultural values, role requirements, and organizational norms, as well as by the specifics of the information received.'' \cite{Heuer2006}
