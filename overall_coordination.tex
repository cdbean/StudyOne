\subsection{Overall Coordination}

Teams had a mixture of collaborative settings. During class teams worked collocated and synchronously. When outside class, 15 teams (60\%) worked remotely, 5 teams collocated, and the other 5 teams had a mix of collocated and distributed work. In terms of time, only one team chose to work asynchronously. When asked how they coordinated work in class and outside class respectively, teams answered they accomplished different things: when in class, they worked closely on things that have high dependency, such as discussing overall direction, negotiating role division, and dividing work into loosely coupled modules. These work prepared them to be able to work on individual modules separately outside class. Teams reflected that it is easier to articulate opinions and analysis with face-to-face interactions as they require high-frequency conversation turns. Especially when team members have conflicted opinions, teams prefer to talk directly to reach an agreement. 

\begin{quote}
	Most of the work done outside of class was planned for and outlined when we could easily talk in class. Then we would continue what we had established while working apart from each other (P78)
\end{quote}

\begin{quote}
	We finished all of our annotations in class and had a good foundation of what we as a group believed went down…Once we had everything annotated through CAnalytics we primarily then worked via this tool and google docs to craft the final submission and bring everything together. We had a plan to meet up outside of class but found with a combination of these two tools we didn’t actually need to. (P148)
\end{quote}

\begin{quote}
	If we did not have CAnalytics, we probably would have had to meet in person. Even if we used Google docs, it would have been hard to coordinate all the work (P126)
\end{quote}


In the survey 88\% participants rated positively on team awareness, indicating that most of them felt aware of what their team members did. We then find that team self-reported awareness was significantly correlated with perceived performance, meaning that when individuals kept updated with other team member's activities, they felt more confident with their team performance. However, no correlation was found between awareness and team's real performance. 

The awareness features were received well. When asked what features helped them stay aware of team activities, 28 participants mentioned the tool coordinator, 24 mentioned the notification system, 19 mentioned the history tool, 14 mentioned the real-time update of user-generated data.  

\begin{quote}
The updates in the corner of the screen and the dot that appeared on the tabs was the primary way that helped me determine what my teammate was working on. For example, when there was a dot in the document tab, I knew that my teammate was working on annotating evidence. Also, if they were in the network tab, I knew that they were drawing inferences from the evidence we had collected. (User 166)
\end{quote}

\begin{quote}
	In CAnalytics there is a history tab where you can check all of the changes that were made and when so that my teammates and I could check the progress of the project and see who exactly was making changes to what.(U151)
\end{quote}

Besides, participants commented that being aware made them more engaged in the work. Being able to see teammate's real-time contribution motivates individuals to make equal or more contributions. 

\begin{quote}
the notifications every time you saw someone annotated something kept you peace at mind that your teammates are also working efficiently on this project. (U64)
\end{quote}
\begin{quote}
During class I wasn’t sure if my teammates were doing work for that class or another thing but then seeing their dot switch between applications on the software and updates pop up on my screen I knew they were doing work for 231. (U141)
\end{quote}
\begin{quote}
CAnalytics allowed us to see how much each person had contributed to their work (142)
\end{quote}

